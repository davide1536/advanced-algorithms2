\section{Algoritmo 2-approssimato}
\label{AlgoritmoApprossimato}

L' \textbf{\textit{algoritmo 2-approssimato}} utilizza come sottoprocedura l'algoritmo di Prim, che fornisce un MST.
L'algoritmo 2-approssimato sfrutta la disugualianza triangolare, ossia dati tre nodi X, Y, Z, la distanza tra X e Y è al più la distanza tra X e Y sommata alla distanza tra Y e Z.  
Questo algoritmo è considerato 2-approssimato in quanto crea un ciclo con un costo che è al massimo due volte il costo della soluzione ottima.

\subsection{Strutture dati}
\label{struttureDati3}

Le strutture dati e i metodi utilizzati per implementare questo algoritmo sono:

\begin{itemize}
    \item classe \hyperlink{subsection.2.1}{Grafo};
    \item classe \hyperlink{subsection.2.2}{Nodo};
    \item metodo \hyperlink{getNodo}{getNodo(id\_nodo)};
    \item \hyperlink{getNodo}{getNodo(id\_nodo)};
    \item metodo \hyperlink{prim}{prim(g, radice)};
    \item metodo \hyperlink{getTree}{getTree(g)};
    \item metodo \hyperlink{preOrderVisit}{preOrderVisit(nodo, h)}.
\end{itemize}


\subsection{Implementazione}
\label{implementazione3}

Questo algoritmo oltre ad essere un algoritmo che fornisce una buona approssimazione della soluzione ottima del problema del commesso viaggiatore e ad avere una buona complessità asintotica consente anche di essere implementato molto semplicemente.
Per poterlo implementare sono stati quindi seguiti i seguenti passi:

\begin{itemize}
    \item grazie alle funzioni \hyperlink{prim}{prim(g, radice)} \hyperlink{getTree}{getTree(g)} è stato calcolato l'albero di copertura minimo del grafo;
    \item successivamente è stata fatta la sua visita anticipata con la funzione \hyperlink{preOrderVisit}{preOrderVisit(nodo, h)};
    \item sia \texttt{h} la lista di nodi ottenuta come output da \hyperlink{preOrderVisit}{preOrderVisit(nodo, h)} per ottenere la soluzione è bastato comporre h con la radice dell'albero (primo elemento di h).
\end{itemize}
