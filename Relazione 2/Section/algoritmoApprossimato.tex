\section{Algoritmo 2-approssimato}
\label{AlgoritmoApprossimato}

L' \textbf{\textit{algoritmo 2-approssimato}} utilizza come sottoprocedura l'algoritmo di Prim, che fornisce un MST.
L'algoritmo 2-approssimato sfrutta la disugualianza triangolare, ossia dati tre nodi X, Y, Z, la distanza tra X e Y è al più la distanza tra X e Y sommata alla distanza tra Y e Z.  
Questo algoritmo è considerato 2-approssimato in quanto crea un ciclo con un costo che è al massimo due volte il costo della soluzione ottima.

\subsection{Strutture dati}
\label{struttureDati3}

Le strutture dati e i metodi utilizzati per implementare questo algoritmo sono:

\begin{itemize}
    \item classe \hyperlink{subsection.2.1}{Grafo};
    \item classe \hyperlink{subsection.2.2}{Nodo};
    \item metodo \hyperlink{getNodo}{getNodo(id\_nodo)};
    \item \hyperlink{getNodo}{getNodo(id\_nodo)};
    \item metodo \hyperlink{prim}{prim(g, radice)};
    \item metodo \hyperlink{getTree}{getTree(g)};
    \item metodo \hyperlink{preOrderVisit}{preOrderVisit{nodo, h}}.
\end{itemize}


\subsection{Implementazione}
\label{implementazione3}

L'algoritmo è implementato nel seguente modo:

\begin{itemize}
    \item viene inizializzata una lista \texttt{h} vuota;
    \item viene assegnata alla variabile \texttt{radice} il primo vertice del grafo \texttt{g} passato in input;
    \item successivamente viene invocato il metodo \hyperlink{prim}{prim(g, radice)} che permette di calcolare il minumum spanning tree del grafo \texttt{g};
    \item utilizzando il metodo \hyperlink{getTree}{getTree(g)} viene identificato l'albero di copertura minimo;
    \item per mezzo del metodo \hyperlink{preOrderVisit}{preOrderVisit(nodo, h)} viene effettuata la visita anticipata di quell'albero e assegnata ad \texttt{h} una lista di nodi ordinati in base alla visita effettuata, alla quale successivamente viene aggiunto il nodo \texttt{radice};
    \item successivamente viene inizializzato l' \texttt{hamiltonCycle} ad \texttt{h};
    \item dopo aver effettuato una conversione nel ciclo hamiltoniano da oggetti ad interi, questo viene restituito.
\end{itemize}
