\section{Euristiche costruttive}
\label{EuristicheCostruttive}

Con il termine \textit{euristiche costruttive} si identifica una ampia famiglia di euristiche che arrivano alla
soluzione procedendo un vertice alla volta seguendo delle regole prefissate. Lo schema generale di
queste euristiche è composto da tre passi:

\begin{enumerate}
    \item \textbf{Inizializzazione}: scelta del ciclo parziale iniziale (o del punto di partenza);
    \item \textbf{Selezione}: scelta del prossimo vertice da inserire nella soluzione parziale;
    \item \textbf{Inserimento}: scelta della posizione dove inserire il nuovo vertice.
\end{enumerate}

In questo progetto si è deciso di implementare l'euristica \textbf{\textit{Closest Insertion}}, la quale è caratterizzata da:

\begin{itemize}
    \item un insieme di vertici $C \subseteq V$ che rappresentano un circuito parziale;
    \item un vertice k non appartenente a C.
\end{itemize}

Con \textit{distanza} di k da C si intende il minimo peso di un arco che collega k a C. Nell'esutistica \textbf{\textit{Closest Insertion}} si vanno a selezionare i vertici che minimizzano la distanza dal circuito parziale.

\subsection{Strutture dati}
\label{struttureDati2}

Le strutture dati utilizzate per implementare questo algoritmo sono:

%\begin{itemize}
%    \item 
%\end{itemize}

Per la descrizione delle classi si rimanda alla sezione \hyperlink{section.2}{1.2 Classi}.
\newline

\subsection{Implementazione}
\label{implementazione2}

L'algoritmo è stato implementato nel seguente modo:
%\begin{itemize}
%    \item 
%\end{itemize}
