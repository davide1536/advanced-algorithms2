\section{Euristiche costruttive}
\label{EuristicheCostruttive}

Con il termine \textit{euristiche costruttive} si identifica una ampia famiglia di euristiche che arrivano alla
soluzione procedendo un vertice alla volta seguendo delle regole prefissate. Lo schema generale di
queste euristiche è composto da tre passi:

\begin{enumerate}
    \item \textbf{Inizializzazione}: scelta del ciclo parziale iniziale (o del punto di partenza);
    \item \textbf{Selezione}: scelta del prossimo vertice da inserire nella soluzione parziale;
    \item \textbf{Inserimento}: scelta della posizione dove inserire il nuovo vertice.
\end{enumerate}

In questo progetto si è deciso di implementare l'euristica \textbf{\textit{Closest Insertion}}, la quale è caratterizzata da:

\begin{itemize}
    \item un insieme di vertici $C \subseteq V$ che rappresentano un circuito parziale;
    \item un vertice k non appartenente a C.
\end{itemize}

Con \textit{distanza} di k da C si intende il minimo peso di un arco che collega k a C. Nell'esutistica \textbf{\textit{Closest Insertion}} si vanno a selezionare i vertici che minimizzano la distanza dal circuito parziale.

\subsection{Strutture dati}
\label{struttureDati2}

Le strutture dati utilizzate per implementare questo algoritmo sono:

\begin{itemize}
    \item classe \hyperlink{subsection.2.1}{Grafo};
    \item classe \hyperlink{subsection.2.2}{Nodo};
    \item metodo \hyperlink{updateCiclo}{updateCiclo(nodo, ciclo, verticiCiclo, noCiclo, position)};
    \item metodo \hyperlink{getClosestNode}{getClosestNode(g, ciclo, noCiclo))};
    \item metodo \hyperlink{getPosition}{getPosition(g, k, ciclo, noCiclo))}.
\end{itemize}


\subsection{Implementazione}
\label{implementazione2}

L'algoritmo è stato implementato nel seguente modo:

\begin{enumerate}

    \item inizializzazione: 

    \begin{itemize}
        \item sono state inizializzate tre liste: una contenente una copia dei nodi presenti nel grafo passato in input e altre due inizializzate vuote, rispettivamente \texttt{verticiCiclo} e \texttt{circuitoParziale};
        \item viene identificato con la variabile \texttt{radice} il primo nodo da inserire all'interno del circuto parziale e utilizzando il metodo \hyperlink{updateCiclo}{updateCiclo(nodo, ciclo, verticiCiclo, noCiclo, position)} viene costruito il circuito parziale;
        \item successivamente viene trovato, utilizzando il metodo \hyperlink{getClosestNode}{getClosestNode(g, ciclo, noCiclo))}, un vertice \texttt{nodo\_k}, che minimizza la distanza con il nodo presente nel circuito parziale e, utilizzando il metodo \hyperlink{updateCiclo}{updateCiclo(nodo, ciclo, verticiCiclo, noCiclo, position)}, viene aggiornato il circuito parziale.
    \end{itemize}

    \item selezione:

    \begin{itemize}
        \item utilizzando l'iterazione sui vertici non ancora presenti nel circuito, viene cercato, utilizzando il metodo \hyperlink{getClosestNode}{getClosestNode(g, ciclo, noCiclo))}, un vertice \texttt{nodo\_k}, che minimizza la distanza tra esso e il circuito parziale.
    \end{itemize}

    \item inserimento:

    \begin{itemize}
        \item utilizzando il metodo \hyperlink{getPosition}{getPosition(g, k, ciclo, noCiclo))} viene cercato l'arco del circuito parziale che minimizza la distanza con il nodo trovato durante il passo di selezione, il \texttt{nodo\_k}, ed infine viene inserito il vertice \texttt{nodo\_k} tra i due nodi dell'arco appena trovati;
    \end{itemize}

    si giunge così ad ottenere un circuito.
    
\end{enumerate}