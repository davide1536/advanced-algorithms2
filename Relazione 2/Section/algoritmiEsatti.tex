\section{Algoritmo esatto di Held e Karp}
\label{algoritmoEsatto}

L' \textbf{\textit{algoritmo esatto di Held e Karp}} è basato sull'utilizzo della programmazione dinamica, tecnica che permette di risolvere problemi di ottimizzazione combinando le soluzioni di sottoproblemi più semplici, per risalire poi alla soluzione cercata.

\subsection{Strutture dati}
\label{struttureDati1}

Le strutture dati utilizzate per implementare questo algoritmo sono:

\begin{itemize}
    \item classe \hyperlink{subsection.2.1}{Grafo};
    \item metodo \hyperlink{hkvisit}{hkVisit(g, v, S, start)};
    \item metodo \hyperlink{hktsp}{hkTsp(g))}.
\end{itemize}

\subsection{Implementazione}
\label{implementazione1}

Questo algoritmo è stato implementato nel seguente modo:

\begin{itemize}
    \item viene eseguita la chiamata a \hyperlink{hktsp}{hkTsp(g))}, che a sua volta chiama \hyperlink{hkvisit}{hkVisit(g,v,S, start)} con input la coppia nodo 0 e sottoinsime la lista di tutti i nodi.
    
    \item inizia quindi l'esecuzione dell'algoritmo vero e proprio;
    
    \item come primo passo viene creato un id univoco per la coppia servita in input all'algoritmo. Per motivi implementativi l'id non è altro che una stringa costruita nel seguente modo:
    
    \begin{itemize}
        \item data una coppia \texttt{(v,S)} --> l'id sarà la stringa \texttt{'[v[S]]'}
    \end{itemize}
    
    in questo modo è possibile ricostruire in qualsiasi momento l'id corrispondente ad una coppia, e trovare in tempo O(1) il peso del circuito relativo, all'interno del dizionario diz\_pesi;
    
    \item si procede quindi con il caso base, cioè quando il sottoinsieme S è composto dal solo nodo v, in questa evenienza si restituisce semplicemente il peso da 0 a v;
    
    \item nel caso in cui, invece, il sottoproblema fosse già stato calcolato in precedenza (tipico della programmazione dinamica), si restituisce il peso corrispondente alla coppia (e quindi al circuito parziale);
    
    \item se non si rientra in uno dei casi precedenti, si procede con il calcolo effettivo del peso del circuito parziale:
    
    \begin{itemize}
        \item vengono inizializzati il peso e il padre del circuito all'interno dei dizionari;
        
        \item per ogni nodo u, presente nel sottoinsieme, si calcolano tutti i sottoproblemi di dimensione più piccola, richiamando quindi \hyperlink{hkvisit}{hkVisit(g, u, S, start)} con input nodo u e sottoinsieme \( S\setminus \left \{ v \right \} \).
        
    \end{itemize}
    
    \item a questo punto si definisce una soluzione parziale e, se la soluzione parziale salvata al passo n-1 contiene meno nodi di quella calcolata al passo n, viene aggiornata;
    
    \item per finire viene eseguito il controllo del timer di 3 minuti e si decide se procedere o lanciare l'eccezione \texttt{HaltException()} ed interrompere l'esecuzione del programma.
    
\end{itemize}

\newpage