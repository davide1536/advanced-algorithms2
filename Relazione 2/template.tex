\documentclass[a4paper, titlepage]{article}
\usepackage[table]{xcolor}
\usepackage[T1]{fontenc}
\usepackage[utf8]{inputenc}
\usepackage[italian]{babel}
\usepackage{amsmath}
\usepackage{listings}
\usepackage{textcomp}
\usepackage{multirow}
\usepackage{multicol}
\usepackage{booktabs}
\usepackage{graphicx}
\usepackage{floatflt}
\usepackage{epsfig}
\usepackage{pstricks}
\usepackage{subfigure}
\usepackage[labelfont=bf, font=scriptsize]{caption}
\usepackage[italian]{varioref}
\usepackage[suftesi]{frontespizio}
\usepackage{color}
\usepackage{tikz}
\usepackage{caption}
\usepackage{comment}
\usepackage{lipsum}
\usepackage{hyperref}
\usepackage{graphicx}
\usepackage[margin=2cm]{geometry}
%\usepackage{pgfplots}
\usepackage{comment}
\usepackage{lipsum}
\usepackage{longtable}
%\pgfplotsset{compat=1.16}

%%% Il documento vero e proprio %%%
\begin{document}

%%% Frontespizio %%%
\begin{frontespizio}
\Universita{Padova} 
\Logo{Fig/logo_unipd} 
\Dipartimento{Matematica}
\Corso[Laurea Magistrale]{Informatica} 

\Annoaccademico{2020-2021}
\Titoletto{Algoritmi Avanzati \\ Relazione di laboratorio 2}
\Titolo{Traveling Salesman Problem}
\Sottotitolo{Maggio 2021}
\NCandidati{Gruppo di lavoro} 
\Candidato[2029254]{Sofia Bononi}
\Candidato[2022146]{Davide Franzoso}
\Candidato[2029214]{Matteo Mariani}
\end{frontespizio}

%%% Indice %%%
\tableofcontents

\newpage
%%% Sezioni %%%
\section{Introduzione}
\label{Introduzione}

In questa relazione vengono descritte l'implementazione e lo studio delle performance dell' \texttt{algoritmo esatto di Held e Karp, l'euristica costruttiva Closest Insertion e l'algoritmo 2-approssimato}.
Questi tre algoritmi vengono utilizzati per risolvere il problema del Travelling Salesman Problem (TSP, problema del commesso viaggiatore). Questo problema si può rappresentare con un grafo non orientato, pesato e
completo G = (V,E), dove il peso dell’arco {u, v} è uguale alla distanza da u a v. Risolvere il TSP significa quindi trovare un circuito Hamiltoniano, ossia un ciclo che visita tutti i vertici esattamente una volta, di costo minimo.
\newline

\textbf{Nota}: per una maggiore leggibilità e una più semplificata analisi del documento OGNI riferimento a metodo/funzione è cliccabile, in modo da rendere accessibile nel minor tempo possibile la rispettiva implementazione e utilità.
\newpage
\section{Classi}
\label{Classi}

Per implementare gli algoritmi sono state create le seguenti classi di oggetti, una classe Utility e un Main:

\subsection{Arco}
\label{arco}

L'oggetto \textit{Arco} contiene le infomazioni relative ad un arco:

\begin{itemize}
    \item \textbf{nodo1}: nodo iniziale;
    \item \textbf{nodo2}: nodo finale;
    \item \textbf{peso}: peso dell'arco;
    \item \hypertarget{getarco}{\textbf{\textsc{getArco()}}}: funzione che restituisce una lista [nodo1, nodo2, peso];
    \item \hypertarget{getarcoinverso}{\textbf{\textsc{getArcoInverso()}}}: funzione che, chiamata su un arco \emph{(u,v)} , ne restituisce l'inverso \emph{(v,u)}.
\end{itemize}

%%%%%%%%%%%%%%%%%%%%%%%%%%%%%%%%%%%%%%%%%%%%%%%%%%%%%%%%%%%%%%%%%%%%%%%%%%%%%%%%%%%%%%%%%%%%%%%%%%%%%%%%%%%%%%%%%%%%%%%%%%%%%%%

\subsection{Grafo}
\label{grafo}

L'oggetto \textit{Grafo} contiene le informazioni relative ad un grafo:

\begin{itemize}
    \item \textbf{n\_nodi}: campo dati che indica il numero dei nodi;
    \item \textbf{name}:;
    \item \textbf{g\_type}:;
    \item \textbf{comment}:;
    \item \textbf{edge\_weigt\_type}:;
    \item \textbf{edge\_weigt\_format}:;
    \item \textbf{display\_data\_type}:;
    \item \textbf{lista\_nodi}: campo dati che contiene un insieme set() di nodi; 
    \item \textbf{lista\_id\_nodi}:;
    \item \textbf{adj\_matrix}:;
    \item \textbf{id2Node}: dizionario avente come campo key l'identificatore di un nodo (intero) e come valore l'oggetto nodo corrispondente a quell'identificativo;
    \item \textbf{totPeso}: campo dati che indica il peso totale degli archi del grafo, calcolato da uno degli algoritmi presentati;
    \item \textbf{diz\_pesi}:;
    \item \textbf{diz\_padri}:;
    \item \textbf{id2NodeSet}:;
    \item \hypertarget{getnodo}{\textbf{\textsc{getNodo(id\_nodo)}}}: metodo che dato in input l'id di un nodo restituisce l'oggetto del nodo corrispondente;
    \item \hypertarget{getnodeset}{\textbf{\textsc{getNodeSet(v,S)}}}:;
    \item \hypertarget{getlistanodi}{\textbf{\textsc{getListaNodi()}}}: metodo che restituisce la lista di  oggetti nodi.
\end{itemize}

%%%%%%%%%%%%%%%%%%%%%%%%%%%%%%%%%%%%%%%%%%%%%%%%%%%%%%%%%%%%%%%%%%%%%%%%%%%%%%%%%%%%%%%%%%%%%%%%%%%%%%%%%%%%%%%%%%%%%%%%%%%%%%%

\subsection{Nodo}
\label{nodo}

L'oggetto \textit{Nodo} contiene le informazioni relative ad un nodo:

\begin{itemize}
    \item \textbf{id}: Intero identificativo del nodo;
    \item \textbf{x}:;
    \item \textbf{y}:;
    \item \hypertarget{inh}{\textbf{in\_h}}: Intero per verificare se un nodo è stato estratto dallo heap o meno;
    \item \textbf{key}: Campo utilizzato per costruire il min-heap, inizializzato con il peso dell'arco adiacente al nodo. Ad esempio il campo key del nodo \emph{u} sarà inizializzato con il valore \emph{w(u,v)};
    \item \textbf{heapIndex}:;
    \item \textbf{padre}: Intero che identifica il padre v di un nodo u nell'albero di copertura minimo;
    \item \textbf{figlio}:;
    \item \hypertarget{hash}{\textbf{\textsc{\_\_hash\_\_()}}}: metodo che restituisce la lista di  oggetti nodi.
\end{itemize}

%%%%%%%%%%%%%%%%%%%%%%%%%%%%%%%%%%%%%%%%%%%%%%%%%%%%%%%%%%%%%%%%%%%%%%%%%%%%%%%%%%%%%%%%%%%%%%%%%%%%%%%%%%%%%%%%%%%%%%%%%%%%%%%

\subsection{NodeSet}
\label{nodeset}

L'oggetto \textit{NodeSet} contiene le seguenti informazioni:

\begin{itemize}
    \item \textbf{v}:;
    \item \textbf{S}:.
\end{itemize}

%%%%%%%%%%%%%%%%%%%%%%%%%%%%%%%%%%%%%%%%%%%%%%%%%%%%%%%%%%%%%%%%%%%%%%%%%%%%%%%%%%%%%%%%%%%%%%%%%%%%%%%%%%%%%%%%%%%%%%%%%%%%%%%

\subsection{Utility}
\label{utility}

La classe \textit{Utility} contiene le funzioni di servizio per implementare i tre algoritmi:

\begin{itemize}
    \item \hypertarget{cehckUniq}{\textbf{\textsc{checkUniq(c)}}}: metodo che controllo l'unicità di ogni nodo all'interno del circuito;
    \item \hypertarget{checkHamiltoCycle}{\textbf{\textsc{checkHamiltoCycle(g, c)}}}: metodo che controlla se il ciclo restituito dalla funzione è hamiltoniano;
    \item hypertarget{computeWeight}{\textbf{\textsc{computeWeight(c, g)}}}: metodo che, dato in input un circuito c e un grafo g, calcola i pesi;
    \item hypertarget{parsing}{\textbf{\textsc{parsing(directory)}}}: metodo che, data in input una directory, analizza tutti i file interni e seleziona solo quelli con estensione .txt, fornendoli in input alla funzione \hyperlink{creagrafi}{crea\_grafi()};
    \item hypertarget{crea\_grafi}{\textbf{\textsc{crea\_grafi(path)}}}: metodo che, dato in input un file, lo legge ed estrae le informazioni per la creazione di oggetti Grafo, Nodo e Arco;
    \item hypertarget{convert}{\textbf{\textsc{convert(x)}}}:;
    \item hypertarget{calcGeoDist}{\textbf{\textsc{calcGeoDist(nodo1, nodo2)}}}:;
    \item hypertarget{calcEuclDist}{\textbf{\textsc{calcEuclDist(nodo1, nodo2)}}}:;
    \item hypertarget{creaSottoinsiemi}{\textbf{\textsc{creaSottoinsiemi(grafo)}}}: metodo utilizzato per inizializzare le coppie v,S;
    \item hypertarget{sub\_seq}{\textbf{\textsc{sub\_seq(grafo, n, i, g, x, s)}}}: metodo che crea le sequenze;
    \item hypertarget{prim}{\textbf{\textsc{prim(g, radice)}}}:;
    \item hypertarget{getTree}{\textbf{\textsc{getTree(g)}}}:;
    \item hypertarget{preOrderVisit}{\textbf{\textsc{preOrderVisit(nodo, h)}}}:;
    \item hypertarget{min\_reloaded}{\textbf{\textsc{min\_reloaded(list, i)}}}: metodo utilizzato per identificare l'algoritmo che restituisce il peso minore;
    \item hypertarget{output\_peso}{\textbf{\textsc{output\_peso(ista\_grafi, peso\_held\_karp, peso\_euristica, peso\_due\_approssimato, tempo\_held\_karp, tempo\_euristica, tempo\_due\_approssimato)}}}: metodo utilizzato per creare la tabella dei risultati.
\end{itemize}

%%%%%%%%%%%%%%%%%%%%%%%%%%%%%%%%%%%%%%%%%%%%%%%%%%%%%%%%%%%%%%%%%%%%%%%%%%%%%%%%%%%%%%%%%%%%%%%%%%%%%%%%%%%%%%%%%%%%%%%%%%%%%%%

\subsection{Heap}
\label{heap}

L'oggetto \textit{Heap} contiene le funzioni che permettono di realizzare un \textit{min-heap}. Uno heap è una struttura dati che si può considerare un albero binario quasi completo, ovvero un albero nel quale tutti i livelli sono riempiti quasi completamente tranne l'ultimo. Le caratteristiche dello heap sono:

\begin{itemize}
    \item l'ultimo livello incompleto si riempie da sinistra;
    \item l'albero viene rappresentato sotto forma di array A che è caratterizzato da \textit{A.length} che rappresenta la lunghezza di A e \textit{A.heapsize} che rappresenta la dimensione dello heap.
\end{itemize}


Per implementare l'algoritmo è stato realizzato un \textit{min-heap}, uno heap caratterizzato da nodi con la chiave minore o uguale a quella dei figli. Quindi la radice rappresenta l'elemento più piccolo del \textit{min-heap}.


\begin{itemize}
    \item \textbf{vector}: vettore associato allo heap;
    \item \textbf{length}: lunghezza del vettore;
    \item \textbf{heapsize}: lunghezza dello heap;
    \item \hypertarget{buildminheap}{\textbf{\textsc{BuildMinHeap(h)}}}: metodo che dato in input l'oggetto heap, costruisce un \textit{min-heap};
    \item \hypertarget{minheapify}{\textbf{\textsc{MinHeapify(h, i)}}}: metodo che dato in input un vettore e un nodo, sistema il nodo nella posizione corretta;
    \item \hypertarget{heapdecreasekey}{\textbf{\textsc{HeapDecreaseKey(h, i ,key)}}}: metodo che dato in input uno heap, un indice ed una nuova chiave, sostituisce il valore del vettore associato all'indice i con il nuovo valore key;
    \item \hypertarget{heapminimum}{\textbf{\textsc{HeapMinimum(h)}}}: metodo che dato in input uno heap, restituisce il valore minimo dello heap, ossia la radice;
    \item \hypertarget{heapextractmin}{\textbf{\textsc{HeapExtractMin(h)}}}: metodo che dato in input uno heap, trova l'elemento più piccolo, lo rimuove e ritorna l'oggetto;
    \item \hypertarget{isin}{\textbf{\textsc{isIn(h, v)}}}: metodo che, dato in input un nodo, restituisce 1 se esso è presente nello heap, 0 altrimenti. Per rendere questa operazione di tempo costante è stata aggiunto un attributo \hyperlink{inh}{\textit{in\_h}} ad ogni nodo;
    \item \hypertarget{right}{\textbf{\textsc{right(index)}}}: metodo che, dato in input un indice di un nodo, restituisce l'indice del nodo figlio destro;
    \item \hypertarget{left}{\textbf{\textsc{left(index)}}}: metodo che, dato in input un indice di un nodo, restituisce l'indice del nodo figlio sinistro;
    \item \hypertarget{parent}{\textbf{\textsc{parent(index)}}}:metodo che, dato in input un indice di un nodo, restituisce l'indice del padre.
\end{itemize}

%%%%%%%%%%%%%%%%%%%%%%%%%%%%%%%%%%%%%%%%%%%%%%%%%%%%%%%%%%%%%%%%%%%%%%%%%%%%%%%%%%%%%%%%%%%%%%%%%%%%%%%%%%%%%%%%%%%%%%%%%%%%%%%

\subsection{Main}
\label{main}

La classe \textit{Main} contiene le funzioni misurazione di performance e le implementazioni degli algoritmi:

\begin{itemize}
    \item \hypertarget{measureRunTime}{\textbf{\textsc{measureRunTime(algorithm)}}}:metodo che calcola i tempi di esecuzione di ciasun algoritmo. Per aumentare l'affidabilità delle misurazioni per le istanze con un numero di nodi <=100, gli algoritmi sono stati ripetuti più volte (30) ed infine il tempo è stato calcolato dividendo il tempo totale per il numero di iterazioni;
    \item \hypertarget{measurePerformance}{\textbf{\textsc{measurePerformance()}}}:.
\end{itemize}


\newpage
\newpage
\section{Algoritmo esatto di Held e Karp}
\label{algoritmoEsatto}

L' \textbf{\textit{algoritmo esatto di Held e Karp}} è basato sull'utilizzo della programmazione dinamica, tecnica che permette di risolvere problemi di ottimizzazione combinando le soluzioni di sottoproblemi più semplici, per risalire poi alla soluzione cercata.

\subsection{Strutture dati}
\label{struttureDati1}

Le strutture dati utilizzate per implementare questo algoritmo sono:

\begin{itemize}
    \item classe \hyperlink{subsection.2.1}{Grafo};
    \item metodo \hyperlink{hkvisit}{hkVisit(g, v, S, start)};
    \item metodo \hyperlink{hktsp}{hkTsp(g))}.
\end{itemize}

\subsection{Implementazione}
\label{implementazione1}

Questo algoritmo è stato implementato nel seguente modo:

\begin{itemize}
    \item viene eseguita la chiamata a \hyperlink{hktsp}{hkTsp(g))}, che a sua volta chiama \hyperlink{hkvisit}{hkVisit(g,v,S, start)} con input la coppia nodo 0 e sottoinsime la lista di tutti i nodi.
    
    \item inizia quindi l'esecuzione dell'algoritmo vero e proprio;
    
    \item come primo passo viene creato un id univoco per la coppia servita in input all'algoritmo. Per motivi implementativi l'id non è altro che una stringa costruita nel seguente modo:
    
    \begin{itemize}
        \item data una coppia \texttt{(v,S)} --> l'id sarà la stringa \texttt{'[v[S]]'}
    \end{itemize}
    
    in questo modo è possibile ricostruire in qualsiasi momento l'id corrispondente ad una coppia, e trovare in tempo O(1) il peso del circuito relativo, all'interno del dizionario diz\_pesi;
    
    \item si procede quindi con il caso base, cioè quando il sottoinsieme S è composto dal solo nodo v, in questa evenienza si restituisce semplicemente il peso da 0 a v;
    
    \item nel caso in cui, invece, il sottoproblema fosse già stato calcolato in precedenza (tipico della programmazione dinamica), si restituisce il peso corrispondente alla coppia (e quindi al circuito parziale);
    
    \item se non si rientra in uno dei casi precedenti, si procede con il calcolo effettivo del peso del circuito parziale:
    
    \begin{itemize}
        \item vengono inizializzati il peso e il padre del circuito all'interno dei dizionari;
        
        \item per ogni nodo u, presente nel sottoinsieme, si calcolano tutti i sottoproblemi di dimensione più piccola, richiamando quindi \hyperlink{hkvisit}{hkVisit(g, u, S, start)} con input nodo u e sottoinsieme \( S\setminus \left \{ v \right \} \).
        
    \end{itemize}
    
    \item a questo punto si definisce una soluzione parziale e, se la soluzione parziale salvata al passo n-1 contiene meno nodi di quella calcolata al passo n, viene aggiornata;
    
    \item per finire viene eseguito il controllo del timer di 3 minuti e si decide se procedere o lanciare l'eccezione \texttt{HaltException()} ed interrompere l'esecuzione del programma.
    
\end{itemize}

\newpage	
\newpage
\section{Euristiche costruttive}
\label{EuristicheCostruttive}

Con il termine \textit{euristiche costruttive} si identifica una ampia famiglia di euristiche che arrivano alla
soluzione procedendo un vertice alla volta seguendo delle regole prefissate. Lo schema generale di
queste euristiche è composto da tre passi:

\begin{enumerate}
    \item \textbf{Inizializzazione}: scelta del ciclo parziale iniziale (o del punto di partenza);
    \item \textbf{Selezione}: scelta del prossimo vertice da inserire nella soluzione parziale;
    \item \textbf{Inserimento}: scelta della posizione dove inserire il nuovo vertice.
\end{enumerate}

In questo progetto si è deciso di implementare l'euristica \textbf{\textit{Closest Insertion}}, la quale è caratterizzata da:

\begin{itemize}
    \item un insieme di vertici $C \subseteq V$ che rappresentano un circuito parziale;
    \item un vertice k non appartenente a C.
\end{itemize}

Con \textit{distanza} di k da C si intende il minimo peso di un arco che collega k a C. Nell'esutistica \textbf{\textit{Closest Insertion}} si vanno a selezionare i vertici che minimizzano la distanza dal circuito parziale.

\subsection{Strutture dati}
\label{struttureDati2}

Le strutture dati utilizzate per implementare questo algoritmo sono:

%\begin{itemize}
%    \item 
%\end{itemize}

Per la descrizione delle classi si rimanda alla sezione \hyperlink{section.2}{1.2 Classi}.
\newline

\subsection{Implementazione}
\label{implementazione2}

L'algoritmo è stato implementato nel seguente modo:
%\begin{itemize}
%    \item 
%\end{itemize}

\newpage
\section{Algoritmo 2-approssimato}
\label{AlgoritmoApprossimato}

L' \textbf{\textit{algoritmo 2-approssimato}} utilizza come sottoprocedura l'algoritmo di Prim, che fornisce un MST.
L'algoritmo 2-approssimato sfrutta la disugualianza triangolare, ossia dati tre nodi X, Y, Z, la distanza tra X e Y è al più la distanza tra X e Y sommata alla distanza tra Y e Z.  
Questo algoritmo è considerato 2-approssimato in quanto crea un ciclo con un costo che è al massimo due volte il costo della soluzione ottima.

\subsection{Strutture dati}
\label{struttureDati3}

Le strutture dati e i metodi utilizzati per implementare questo algoritmo sono:

\begin{itemize}
    \item classe \hyperlink{subsection.2.1}{Grafo};
    \item classe \hyperlink{subsection.2.2}{Nodo};
    \item metodo \hyperlink{getNodo}{getNodo(id\_nodo)};
    \item \hyperlink{getNodo}{getNodo(id\_nodo)};
    \item metodo \hyperlink{prim}{prim(g, radice)};
    \item metodo \hyperlink{getTree}{getTree(g)};
    \item metodo \hyperlink{preOrderVisit}{preOrderVisit(nodo, h)}.
\end{itemize}


\subsection{Implementazione}
\label{implementazione3}

Questo algoritmo oltre ad essere un algoritmo che fornisce una buona approssimazione della soluzione ottima del problema del commesso viaggiatore e ad avere una buona complessità asintotica consente anche di essere implementato molto semplicemente.
Per poterlo implementare sono stati quindi seguiti i seguenti passi:

\begin{itemize}
    \item grazie alle funzioni \hyperlink{prim}{prim(g, radice)} \hyperlink{getTree}{getTree(g)} è stato calcolato l'albero di copertura minimo del grafo;
    \item successivamente è stata fatta la sua visita anticipata con la funzione \hyperlink{preOrderVisit}{preOrderVisit(nodo, h)};
    \item sia \texttt{h} la lista di nodi ottenuta come output da \hyperlink{preOrderVisit}{preOrderVisit(nodo, h)} per ottenere la soluzione è bastato comporre h con la radice dell'albero (primo elemento di h).
\end{itemize}

%\newpage
%\input{Section/correttezza.tex}
%\newpage
%\input{Section/performance.tex}
%\newpage
%\input{Section/conclusioni.tex}

\end{document}